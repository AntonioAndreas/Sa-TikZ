% to allow compression of cross references. 
% both pgfmanual and pgfplots manual would be MUCH larger without it.
\pdfminorversion=5 
\pdfobjcompresslevel=2
\documentclass{ltxdoc} % the pgf manual styles are based on ltxdoc
\usepackage[T1]{fontenc}
\usepackage[utf8]{inputenc}
\usepackage{amsmath}
\usepackage{lmodern}

%% Use the tikz package and loading the libraries
\usepackage{tikz}
\usetikzlibrary{calc,positioning,decorations.pathreplacing,switching_architectures,}

%% Loading only the package
%\usepackage{sa-tikz}

\usepackage{calc}
\usepackage{imakeidx}

\usepackage{hyperref}
    \hypersetup{%
        colorlinks=true,
        linkcolor=blue,
        filecolor=blue,
        urlcolor=blue,
        citecolor=blue,
        pdfborder=0 0 0,
    }

% dedicated commands:
\newcommand\Tikz{Ti\textit kZ}
\newcommand{\saTikZ}{\textsc S\textit a-\Tikz}
\newcommand{\library}[1]{}

\def\pgfautoxrefs{1}
\input ./macros/pgfmanual-en-macros.tex

\makeatletter
\def\index@prologue{%
\section*{Index}\addcontentsline{toc}{section}{Index}
}
\makeatother

\usepackage{pgfmanual}

\pgfkeys{
    /pdflinks/search key prefixes in={/tikz/,/pgf/},
    /pdflinks/internal link prefix=pgfp,
    /pdflinks/codeexample links=true,
    /pdflinks/warnings=false,
    /pdflinks/show labels=false,
}
\makeindex
\newcommand{\version}{0.2a}
\newcommand{\versiondate}{December 14, 2012}

\title{\saTikZ\footnote{This package has version number \textit{v}\version\ of \versiondate; it is released under and subject to the \href{http://www.latex-project.org/lppl/}{\LaTeX\ Project Public License (LPPL)}.}}
\author{Claudio Fiandrino \\ \small\href{mailto:claudio.fiandrino@gmail.com}{\texttt{claudio.fiandrino@gmail.com}}}
\date{\versiondate}

\begin{document}
\maketitle
\tableofcontents

\section*{Introduction}
\addcontentsline{toc}{section}{Introduction}
The \saTikZ\ library helps in drawing \emph{switching-architectures}. In particular, one of its aims, is to help students to verify if their exercises are correct, but it could also help teachers in preparing lecture notes.

The \saTikZ\ library can be loaded by means of:
\begin{flushleft}
\verb|\usetikzlibrary{|\bgroup\color{red!75!black}\verb|switching_architectures|\egroup\verb|}|
\end{flushleft}
and in this case you should also load manually:
\begin{flushleft}
\verb|\usepackage{|\bgroup\color{red!75!black}\verb|tikz|\egroup\verb|}|\\
\verb|\usetikzlibrary{|\bgroup\color{red!75!black}\verb|calc, positioning, decorations.pathreplacing|\egroup\verb|}|
\end{flushleft}
or by means of:
\begin{flushleft}
\verb|\usepackage{|\bgroup\color{red!75!black}\verb|sa-tikz|\egroup\verb|}|
\end{flushleft}
In the latter case automatically the \Tikz\ package and the libraries \bgroup\color{red!75!black}\verb|calc|\egroup{}, \bgroup\color{red!75!black}\verb|positioning|\egroup\ and \bgroup\color{red!75!black}\verb|decorations.pathreplacing|\egroup\ are loaded.

The version \textit{v}\version\ provides a way to define Clos Networks Strictly-non-Blocking (snb) and Rearrangeable (rear). Future implementations will provide a way to draw also Benes and Cantor Networks.

\section{Basic usage}
The simplest use of the package is to define a 
\begin{command}{{\node}}
    Basic command definition.
\end{command}
with one of the following options
\begin{key}{/tikz/clos snb}
    Option for drawing a Clos Network Strictly-non-Blocking.
\end{key}
\begin{key}{/tikz/clos rear}
    Option for drawing a Clos Network Rearrangeable.
\end{key}
\begin{key}{/tikz/benes}
    Option for drawing a Benes Network.
\end{key}
inside a |tikzpicture| environment:
\begin{environment}{{tikzpicture}\opt{\oarg{options}}}
\end{environment}

\subsection{Examples of Clos Networks}

The following example shows a Rearrangeable Clos Network.

\begin{codeexample}[]
\begin{tikzpicture}
    \node[clos rear] {};
\end{tikzpicture}
\end{codeexample}

The following example shows a Strictly-non-Blocking Clos Network.

\begin{codeexample}[]
\begin{tikzpicture}
    \node[clos snb] {};
\end{tikzpicture}
\end{codeexample}

Notice from the examples that automatically the library is able to compute the constraints that define a Clos Network to be Strictly-non-Blocking or Rearrangeable. Moreover, the network drawn is characterized by:
\begin{itemize}
\item the first stage with:
\begin{itemize}
\item a number of modules equal to 5;
\item each one with two input ports;
\end{itemize}
\item the last stage with:
\begin{itemize}
\item a number of modules equal to 5;
\item each one with two output ports.
\end{itemize}
\end{itemize}
Each module of the network is numbered according to the stage it belongs to.

\subsection{Example of Benes Network}

The simplest example of a Benes Network:
\begin{codeexample}[]
\begin{tikzpicture}
    \node[benes] {};
\end{tikzpicture}
\end{codeexample}
is a Benes Network in which there are 8 input and output ports. 

\begin{codeexample}[]
\begin{tikzpicture}
    \node[benes complete] {};
\end{tikzpicture}
\end{codeexample}

\section{The options}
\subsection{Designing choices - Clos Network}
The two first important design parameters are the total number of input ports of the first stage and the total number of output ports of the last stage. These two parameters could be modified by means of:
\begin{key}{/tikz/N=\marg{value} (initially 10)}
    This is the number of total input ports in the first stage.
\end{key}

\begin{key}{/tikz/M=\marg{value} (initially 10)}
    This is the number of total output ports in the last stage.
\end{key}
Usually, a second design parameter is the number of modules present in the first and last stage. \saTikZ\ defines:
\begin{key}{/tikz/r1=\marg{value} (initially 5)}
    This is the number of total input ports in the first stage.
\end{key}

\begin{key}{/tikz/r3=\marg{value} (initially 5)}
    This is the number of total output ports in the last stage.
\end{key}
The two design parameters provide the number of ports of each module:
\[\textrm{m}_1=\dfrac{N}{r1} \hspace*{2cm} \textrm{m}_3=\dfrac{M}{r3} \]

Some examples considering |N|=9, |r1|=3, |M|=9 and |r3|=3.
\begin{codeexample}[]
\begin{tikzpicture}
    \node[N=9,r1=3,M=9,r3=3,clos rear] {};
\end{tikzpicture}
\end{codeexample}

\begin{codeexample}[]
\begin{tikzpicture}
    \node[N=9,r1=3,M=9,r3=3,clos snb] {};
\end{tikzpicture}
\end{codeexample}

Notice a very important thing: the type of the architecture should be loaded \emph{after} all the design choices when they have been set in the \cs{node}; indeed, if you do not respect this constraint you will end up with an architecture with default values. For example:

\begin{codeexample}[]
\begin{tikzpicture}
    \node[clos rear,N=9,r1=3,M=9,r3=3] {};
\end{tikzpicture}
\end{codeexample}

\subsection{Designing choices - Benes Network}
Benes Networks are composed of $2 \times 2$ modules, so as design choice it just possible to select which is the number of input/output ports:
\begin{key}{/tikz/P=\marg{value} (initially 8)}
    This is the number of total input/output ports in the first/third stage.
\end{key}

Notice that |P| could assume values
\[P=2^p \qquad p=2,3,4,\ldots\]
and the user is responsible to correctly set this parameter.

For low values of $p$ there are no problems in visualizing the network, but as $p$ increases the user should take care of the dimension of the modules and the separation (vertical and horizontal) of the modules: it could be customized as explained in subsection \ref{subsec:customization}.

Here is an example of Benes Network with |P|=16:
\begin{codeexample}[]
\begin{tikzpicture}
    \node[P=16,benes] {};
\end{tikzpicture}
\end{codeexample}

It holds the same thing already said for Clos Networks: set the parameter |P| before declaring the \cs{node} be a |Benes| Network.


\subsection{Output customization}
\label{subsec:customization}
This subsection focuses on how to customize the aspect of the drawing.

\begin{key}{/tikz/module size=\marg{value} (initially 1cm)}
    This option allows to set the module dimension.
\end{key}

\begin{key}{/tikz/module ysep=\marg{value} (initially 1.5)}
    This option allows to set the vertical module distance factor.
\end{key}

\begin{key}{/tikz/module xsep=\marg{value} (initially 3)}
    This option allows to set the horizontal module distance factor.
\end{key}

\begin{key}{/tikz/module label opacity=\marg{value} (initially 1)}
    This option allows to mask the module label when the \meta{value} is set to 0.
\end{key}

\begin{key}{/tikz/pin length factor=\marg{value} (initially 1)}
    This option allows to reduce/increase the length of the pins drawn in input/output. Use a \meta{value} [0,1] to reduce the length or, viceversa, a \meta{value} greater than 1 to increase the length.
\end{key}

The following example shows a Rearrangeable Clos Network with some options modified. Notice that the |module label opacity| should be given as parameter of the desired network.

\begin{codeexample}[]
\begin{tikzpicture}[N=9,r1=3,M=9,r3=3]
    \node[module size=0.5cm,pin length factor=0.5,
        module ysep=1, module xsep=1.25,
        clos rear={module label opacity=0}] {};
\end{tikzpicture}
\end{codeexample}

The options could also be introduced with the standard \Tikz\ syntax:
\begin{command}{{\tikzset}\marg{options}}
    Command that process the \meta{options}: they should be provided separated by a comma.
\end{command}
Therefore, the previous example could be modified into:
\begin{codeexample}[]
\tikzset{module size=0.5cm,pin length factor=0.5,
         module ysep=1, module xsep=1.25,}
\begin{tikzpicture}[N=9,r1=3,M=9,r3=3]
    \node[clos rear={module label opacity=0}] {};
\end{tikzpicture}
\end{codeexample}
An example of Benes Network $32 \times 32$:
\begin{codeexample}[]
\tikzset{module size=0.6cm,pin length factor=0.6,
         module ysep=0.9, module xsep=1.7,}
\begin{tikzpicture}[P=32]
    \node[benes] {};
\end{tikzpicture}
\end{codeexample}

\section{Advanced usage}
In this section some more advanced examples are shown.

\subsection{Identifying front input/output ports}
In this subsection it is shown how to reference the front input and output ports for the first and last stage. Each front input port could be accessed by means of:
\begin{flushleft}
\verb|r1-|\bgroup\color{red!75!black}\verb|module number|\egroup\verb|-|\bgroup\color{red!75!black}\verb|front input|\egroup\verb|-|\bgroup\color{red!75!black}\verb|port number|\egroup; example: \verb|r1-1-front input-1|;
\end{flushleft}
Each front output port could be accessed by means of:
\begin{flushleft}
\verb|r3-|\bgroup\color{red!75!black}\verb|module number|\egroup\verb|-|\bgroup\color{red!75!black}\verb|front output|\egroup\verb|-|\bgroup\color{red!75!black}\verb|port number|\egroup; example: \verb|r3-1-front output-1|;
\end{flushleft}

A simple example with a Rearrangeable Clos network of 4 input and output ports; the first stage and the last one have both 2 modules.

\begin{codeexample}[]
\begin{tikzpicture}[module xsep=2]
  \node[N=4,r1=2,M=4,r3=2,clos rear={module label opacity=0}] {};
  \foreach \name 
    in {r1-1-front input-1,r1-1-front input-2,
        r1-2-front input-1,r1-2-front input-2}
    \node[left] at (\name) {\scriptsize{\texttt{\name}}};
  \foreach \name 
    in {r3-1-front output-1,r3-1-front output-2,
        r3-2-front output-1,r3-2-front output-2}
    \node[right] at (\name) {\scriptsize{\texttt{\name}}};
\end{tikzpicture}
\end{codeexample}

The following is a Strictly-non-Blocking Clos network of 9 input and output ports in which the first and last stage have 3 modules each one.

\begin{codeexample}[]
\begin{tikzpicture}
    \node[N=9,r1=3,M=9,r3=3,clos snb={module label opacity=0}] {};

	\foreach \startmodule in {1,...,3}{
		\foreach \port in {1,...,3}
				 \node[left] at (r1-\startmodule-front input-\port)
				 {\scriptsize{input \startmodule-\port}}; 
	}
	\foreach \startmodule in {1,...,3}{
		\foreach \port in {1,...,3}
				 \node[right] at (r3-\startmodule-front output-\port)
				 {\scriptsize{output \startmodule-\port}}; 
	}    
\end{tikzpicture}
\end{codeexample}

The same applies also for Benes Networks:
\begin{codeexample}[]
\begin{tikzpicture}
    \node[benes={module label opacity=0}] {};

	\foreach \startmodule in {1,...,4}{
		\foreach \port in {1,...,2}
				 \node[left] at (r1-\startmodule-front input-\port)
				 {\scriptsize{input \startmodule-\port}}; 
	}
	\foreach \startmodule in {1,...,4}{
		\foreach \port in {1,...,2}
				 \node[right] at (r3-\startmodule-front output-\port)
				 {\scriptsize{output \startmodule-\port}}; 
	}    
\end{tikzpicture}
\end{codeexample}


\subsection{Identifying input/output ports per module}

It is also possible to access, for each module of each stage, its input and output ports. The syntax is similar to the one used for the front input and output ports; each input port could be accessed by means of:
\begin{flushleft}
\verb|r|\bgroup\color{red!75!black}\verb|stage number|\egroup\verb|-|\bgroup\color{red!75!black}\verb|module number|\egroup\verb|-|\bgroup\color{red!75!black}\verb|input|\egroup\verb|-|\bgroup\color{red!75!black}\verb|port number|\egroup; example: \verb|r1-1-input-1|;
\end{flushleft}
Each output port could be accessed by means of:
\begin{flushleft}
\verb|r|\bgroup\color{red!75!black}\verb|stage number|\egroup\verb|-|\bgroup\color{red!75!black}\verb|module number|\egroup\verb|-|\bgroup\color{red!75!black}\verb|front output|\egroup\verb|-|\bgroup\color{red!75!black}\verb|port number|\egroup; example: \verb|r2-1-output-1|;
\end{flushleft}
This allows to derive connections from the first stage to the last stage. Here is an example. 

\begin{codeexample}[]
\begin{tikzpicture}
    \node[N=8,r1=4,M=8,r3=4,clos rear={module label opacity=0}] {};
    \draw[red,ultra thick](r1-2-input-1)--(r1-2-output-2) 
    	(r2-2-input-2)--(r2-2-output-3)
    	(r3-3-input-2)--(r3-3-output-2);
   	\draw[red,ultra thick](r1-4-input-1)--(r1-4-output-1) 
    	(r2-1-input-4)--(r2-1-output-1)
    	(r3-1-input-1)--(r3-1-output-2);
\end{tikzpicture}
\end{codeexample}
Similarly, in a Benes Network:

\begin{codeexample}[]
\begin{tikzpicture}
    \node[benes={module label opacity=0}] {};
    \draw[red,ultra thick](r1-2-input-1)--(r1-2-output-2) 
    	(r2-2-input-2)--(r2-2-output-3)
    	(r3-3-input-2)--(r3-3-output-2);
   	\draw[red,ultra thick](r1-4-input-1)--(r1-4-output-1) 
    	(r2-1-input-4)--(r2-1-output-1)
    	(r3-1-input-1)--(r3-1-output-2);
\end{tikzpicture}
\end{codeexample}



% In the next version it will be implemented an automatic way to design connection paths.
% Magari metterli in background per non disturbare eventuali label

\section{Didactic purposes}
To quickly draw a Clos Network  it is possible to exploit:
\begin{key}{/tikz/clos snb example}
    Option for quickly drawing a Clos Network Strictly-non-Blocking.
\end{key}
\begin{key}{/tikz/clos rear example}
    Option for quickly drawing a Clos Network Rearrangeable.
\end{key}
In this way the network is not seen in its whole complexity, but it is synthetically  depicted. An example of a Strictly-non-Blocking Clos Network drawn with this approach:
\begin{codeexample}[]
\begin{tikzpicture}[N=8,r1=4,M=8,r3=4]
    \node[clos snb example] {};
\end{tikzpicture}
\end{codeexample}

Similarly, an example of a Rearrangeable Clos Network:
\begin{codeexample}[]
\begin{tikzpicture}[N=8,r1=4,M=8,r3=4]
    \node[clos rear example] {};
\end{tikzpicture}
\end{codeexample}

The networks drawn, automatically display the values at which the input parameters |N|, |M|, |r1| and |r3| have been set. However, to let the user to have the possibility of deploying labels rather than the input parameter values, the following option is available:

\begin{key}{/tikz/clos example with labels}
    Option for quickly drawing a Clos Network with custom labels.
\end{key}

The labels could be customized by means of:
\begin{key}{/tikz/N label=\marg{value} (default N)}
    This options sets the label representing the total number of ports in the first stage.
\end{key}

\begin{key}{/tikz/r1 label=\marg{value} (default r$_1$)}
    This options sets the label representing the number of modules in the first stage.
\end{key}

\begin{key}{/tikz/m1 label=\marg{value} (default m$_1$)}
    This options sets the label representing the number of ports per module in the first stage.
\end{key}

\begin{key}{/tikz/r2 label=\marg{value} (default r$_2$)}
    This options sets the label representing the number of modules in the second stage.
\end{key}

\begin{key}{/tikz/M label=\marg{value} (default M)}
    This options sets the label representing the total number of ports in the last stage.
\end{key}

\begin{key}{/tikz/r3 label=\marg{value} (default r$_3$)}
    This options sets the label representing the number of modules in the last stage.
\end{key}

\begin{key}{/tikz/m3 label=\marg{value} (default m$_3$)}
    This options sets the label representing the number of ports per module in the last stage.
\end{key}

An example with the default values:

\begin{codeexample}[]
\begin{tikzpicture}[N=8,r1=4,M=8,r3=4]
    \node[clos example with labels] {};
\end{tikzpicture}
\end{codeexample}

For automatically have labels in math mode, use:
\begin{key}{/tikz/set math mode labels=\mchoice{true,false} (default false)}
	This option is normally disabled |set math mode labels/.default=false|; to ensure labels be set completely in math mode is sufficient set |set math mode labels=true| before the type of the network.
\end{key}

An example with labels in math mode:
\begin{codeexample}[]
\begin{tikzpicture}[N=8,r1=4,M=8,r3=4, set math mode labels=true]
    \node[clos example with labels] {};
\end{tikzpicture}
\end{codeexample}

Here is an example with custom labels introduced by means of the |\tikzset| syntax.

\begin{codeexample}[]
\tikzset{N label={p$_1$ $\times$ q$_1$},M label={p$_3$ $\times$ q$_3$}, 
r1 label=p$_1$, m1 label=q$_1$, r2 label=p$_2$,r3 label=p$_3$, m3 label=q$_3$}
\begin{tikzpicture}[N=8,r1=4,M=8,r3=4]
    \node[clos example with labels] {};
\end{tikzpicture}
\end{codeexample}

% Verificare che module ysep funzioni anche qui!!!!


% * * * * * * * * * * * * * * * * * * * * * * * * * * 
% belongs to \usepackage{makeindex}
\printindex
\end{document}
